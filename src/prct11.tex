\documentclass{beamer}
\usepackage[utf8]{inputenc}
\usepackage{graphicx}
\newtheorem{definicion}{Definición}

%%%%%%%%%%%%%%%%%%%%%%%%%%%%%%%%%%%%%%%%%%%%%%%%%%%%%%%%%%%%%%%%%%%%%%%%%%%%%%%
\title[Presentación con Beamer]{El numero $\Pi$ }
\author{Cirilo Fleitas Rufino}
\date[25-04-2014]{25 de Abril de 2014}

\usetheme{Madrid}
%\usetheme{Antibes}
%\usetheme{tree}
%\usetheme{classic}

%%%%%%%%%%%%%%%%%%%%%%%%%%%%%%%%%%%%%%%%%%%%%%%%%%%%%%%%%%%%%%%%%%%%%%%%%%%%%%%
\begin{document}
  
%++++++++++++++++++++++++++++++++++++++++++++++++++++++++++++++++++++++++++++++  
\begin{frame}

  %\includegraphics[width=0.15\textwidth]{img/ullesc}
  %\hspace*{7.0cm}
  %\includegraphics[width=0.16\textwidth]{img/fmatesc}
  \titlepage

  \begin{small}
    \begin{center}
     Facultad de Matemáticas \\
     Universidad de La Laguna \\
     Tecnicas Experimentales
    \end{center}
  \end{small}

\end{frame}
%++++++++++++++++++++++++++++++++++++++++++++++++++++++++++++++++++++++++++++++  

%++++++++++++++++++++++++++++++++++++++++++++++++++++++++++++++++++++++++++++++  
\begin{frame}
  \frametitle{Índice}  
  \tableofcontents[pausesections]
\end{frame}
%++++++++++++++++++++++++++++++++++++++++++++++++++++++++++++++++++++++++++++++  


\section{Primera Sección: Definición del numero $\Pi$}



%++++++++++++++++++++++++++++++++++++++++++++++++++++++++++++++++++++++++++++++  
\begin{frame}

\frametitle{Primera Sección}

\begin{definicion}
$\Pi$ es la relación entre la longitud de una circunferencia y su diámetro, en geometría euclidiana. Es un número irracional y una de las constantes matemáticas más importantes. Se emplea frecuentemente en matemáticas, física e ingeniería.El valor numérico de $\Pi$, truncado a sus primeras cifras, es el siguiente:

 $\Pi = 3,14159265358979323846... $ 
    Vease \alert{El numero $\Pi$}~\cite{plan} 

\end{definicion}

\end{frame}
%++++++++++++++++++++++++++++++++++++++++++++++++++++++++++++++++++++++++++++++  

\section{Segunda Sección:Métodos de calculo del el número $\Pi$}

%++++++++++++++++++++++++++++++++++++++++++++++++++++++++++++++++++++++++++++++  
\begin{frame}

\frametitle{Segunda Sección:Método de calculo}
Uno de los métodos de averiguar el valor de $\Pi$ es calcular el perímetro de un polígono de muchos lados que está dentro de un círculo de diámetro conocido,cuantos mas lados tenga el polígono, más se parecerá a la circunferencia, y su perímetro se acercara más a la longitud de la circunferencia.
Vease \alert{Aproximación}~\cite{latex} 


\end{frame}
%++++++++++++++++++++++++++++++++++++++++++++++++++++++++++++++++++++++++++++++  
\section{Curiosidades del numero $\Pi$}
\begin{frame}
\frametitle{Curiosidades}
La probabilidad de que dos enteros positivos escogidos al azar sean primos entre si es $6/(\Pi)^2$.
\end{frame}
%-----------------------------------------------------------------------------
\begin{frame}
la función de densidad de probabilidad para la distribución de Cauchy (estándar)

    $f(x) = 1/(\Pi (1 + x^2))$. 
\end{frame}
\begin{frame}
Gracias a Euler tenemos:
    $e^{i \pi} + 1 = 0$ 
y edemas:    
    $e^{iJ} = cos (J) + isin (J)$\\
    donde i es la unidad imaginaria.
\end{frame}
%++++++++++++++++++++++++++++++++++++++++++++++++++++++++++++++++++++++++++++++  

\section{Bibliografía}
%++++++++++++++++++++++++++++++++++++++++++++++++++++++++++++++++++++++++++++++  
\begin{frame}
  \frametitle{Bibliografía}

  \begin{thebibliography}{10}

    \beamertemplatebookbibitems
    \bibitem{plan}  
    El numero $\Pi{}$ 
    {\small $http://http://es.wikipedia.org/wiki/Numero \Pi$}

    \beamertemplatebookbibitems
    \bibitem[URL: Sobre el numero $\Pi{}$]{latex} 
   Aproximación {\small $http://www.juegosdelogica.com/numero_pi.htm$}
  \end{thebibliography}
\end{frame}

%++++++++++++++++++++++++++++++++++++++++++++++++++++++++++++++++++++++++++++++  
\end{document}
